\documentclass{article}

\title{Project Writeup}
\author{Raymone Miller}
\begin{document}
  \maketitle

  \section{Project Summary}
  Some server daemons will offer services to those that request it, such as
  NTP. Write a math server that will provide similar services to those that
  request a few specific items. The server should accept UDP requests in one
  of three forms.
  \begin{verbatim}
    Fnumber
        Given a decimal number is between 0-300 (inclusive), the response
        packet should be F(number) in hexadecimal, where F() is the Fibonacci
        function.
    Dnumber
        Given a decimal number between 0-10^19 (inclusive), the response packet
        should be that number in hexadecimal.
    Rnvmber
        Given a Roman numeral nvmber between I-MMMM (inclusive), the response
        packet should be that number in hexadecimal.
  \end{verbatim}

  \section{Challenges}
  A challenge I had for this project was getting the fibonacci assembly code
  to work within C. The code was originally written all within main, so I had
  to rewrite parts of it to fit it into a callable function. I also originally
  didn't properly follow the calling conventions for preserving registers.
  This caused it to segfault in ways that I previously did not noticed because
  it was being called instead of just being run directly in main. I took me a
  while to understand this.

  \section{Successes}
  A success I found, which is tied to my challenge, is how I was able to
  convert my fibonacci assembly code into a function and call it from the C
  program. As I stated, it was something I was never able to do before, and I
  am usually not comfortable with assembly. Being able to strip the code I had,
  make the necessary changes, and additions to make the code work felt very
  fulfilling for me.

  \section{Lessons Learned}
  A lesson I learned is that logging in Linux is a lot simpler than I thought.
  I've never used the syslog functions before, and it was surprisingly a lot
  more straightforward that I imagined to be able to log a message to the
  system. And I can easily format the message like a call to printf.

\end{document}
