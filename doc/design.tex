\documentclass[12pt]{article}
\usepackage[english]{babel}
\usepackage[utf8x]{inputenc}
\usepackage{amsmath}
\usepackage{listings}
\usepackage{graphicx}
\usepackage[colorinlistoftodos]{todonotes}

\begin{document}
	\begin{titlepage}
	%https://www.overleaf.com/project/63fcda2d47db2df20619255b
	\newcommand{\HRule}{\rule{\linewidth}{0.5mm}} % Defines a new command for the horizontal lines, change thickness here

	\center % Center everything on the page

	%----------------------------------------------------------------------------------------
	%	HEADING SECTIONS
	%----------------------------------------------------------------------------------------

	\textsc{\Large University of Maryland Baltimore Campus}\\[1.5cm] % Name of your university/college
	\textsc{\Large Tool Developer Qualification Course}\\[0.5cm] % Major heading such as course name
	\textsc{\large James Viner, Raymone Miller}\\[0.5cm] % Minor heading such as course title

	%----------------------------------------------------------------------------------------
	%	TITLE SECTION
	%----------------------------------------------------------------------------------------

	\HRule \\[0.4cm]
		\huge \bfseries FDR\\[0.4cm] % Title of your document
	\HRule \\[1.5cm]
	%----------------------------------------------------------------------------------------
	%	AUTHOR SECTION
	%----------------------------------------------------------------------------------------
	%https://www.overleaf.com/project/63fcda2d47db2df20619255b
	\end{titlepage}
	\tableofcontents
	\newpage
	\section{Project Summary}
		Some server daemons will offer services to those that request
		it, such as NTP. Write a math server that will provide similar
		services to those that request a few specific items. The server
		should accept UDP requests in one of three forms.
		\begin{lstlisting}
Fnumber
    Given a decimal number is between 0-300 (inclusive), the response packet
    should be F(number) in hexadecimal, where F() is the Fibonacci function.
Dnumber
    Given a decimal number between 0-1019 (inclusive), the response packet
    should be that number in hexadecimal.
Rnvmber
    Given a Roman numeral nvmber between I-MMMM (inclusive), the response
    packet should be that number in hexadecimal.
		\end{lstlisting}
	\section{Features Targeted}
	\begin{itemize}
		\item Use Tex to write documentation
		\item Add logging with syslog(3)
		\item Add command-line flag -e
		\item Add command-line flag -i
	\end{itemize}
	\section{Architecture}
		\subsection{Data}
			\begin{lstlisting}[language=C]
typedef struct {
	const struct sockaddr *client;
	socklen_t client_sz;
	char *input;
	char *output;
	size_t input_len;
	size_t output_len;
} request_t;
			\end{lstlisting}
	\section{Significant Functions}
		\begin{lstlisting}
void *service_thread(void *arg);
		\end{lstlisting}
		process thread.

		\begin{lstlisting}
void serve_port(int sd);
		\end{lstlisting}
		process data sent to port, and determine which operation to perform.

		\begin{lstlisting}
int fibonacci(request_t *request);
		\end{lstlisting}
		calculate Fibonacci number in hex up to given input number. Returns error code

		\begin{lstlisting}
int large_hex(request_t *request);
		\end{lstlisting}
		Convert a possibly large number into hex. Returns error code

		\begin{lstlisting}
int roman(request_t *request);
		\end{lstlisting}
		Convert a roman numeral to hex. Returns error code

	\section{Order of features}
		\begin{enumerate}
			\item Initialize a Gitlab page for the project.
			\item Create design plan.
			\item Create server code
			\item Validate user input
			\item Design processing of fibonacci request
			\item Design processing of big hex request
			\item Design processing of roman numeral request
			\item Test output from server to client
			\item Add feature: logging
			\item Add feature: -e error messages
			\item Add feature: -i case matching
			\item Final write up.
		\end{enumerate}
	\section{User Interface}
		When a client connects to the server and sends a request via
		text, the server will send a response back based on the
		request. If the -e flag is used, an error message will be sent
		in response on improper input. Otherwise, no response will
		be sent back on improper input.
\end{document}
